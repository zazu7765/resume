%! TEX root = resume.tex
\documentclass[12pt, letterpaper]{article}
\usepackage[utf8]{inputenc}
\usepackage{titlesec}
\usepackage{geometry}
\usepackage{titling}
\usepackage{hyperref}
\usepackage{xcolor}
\usepackage{ifthen}
\usepackage[parfill]{parskip}
% page layout packages
% optional serif font family rebind
% \renewcommand{\familydefault}{\sfdefault}
\pagestyle{empty} % remove page numbers
% geometry layout setup
\geometry{ 
         letterpaper,
         left=.5in,
         top=.5in,
         bottom=.5in,
         right=.5in
    }
% Hyperlink layout setup
\hypersetup{ 
    colorlinks=true,
    linkcolor=blue,
    urlcolor=black,
    pdfpagemode=FullScreen,
}
% title formatting
\titleformat{\section}{\huge\bfseries}{}{0em}{}[\titlerule]
\titleformat{\subsection}{\bfseries\Large}{}{0em}{}[]
\titleformat{\subsubsection}[runin]{\bfseries}{}{0em}{}
% title section parameters
\author{Daniel Ibanescu}
\date{April 2023}
\renewcommand{\maketitle}{
              \begin{center}
                  {\huge\bfseries\theauthor} \\ % Renders your name
                  daniel\_ibanescu@hotmail.com | % Email
                  \href{https://github.com/zazu7765}{Github} % Github URL
                  | \href{https://linkedin.com/in/ibada}{LinkedIn} % Linkedin URL
              \end{center}
              \vspace{-1.5em} % Add some whitespace
          }
% 4 args - company, year(s), title (optional) and location 
\newenvironment{customS}[4]{ 
% Bold company name and date on the right
 {\bf #1} \hfill {#2} 
% If the third argument is not specified, don't print the job title and location line
 \ifthenelse{\equal{#3}{}}{}{ 
  \\
% Italic styling on job title and location
  {\em #3} \hfill {\em #4} 
  }
% \cdot used for bullets, no indentation
  \begin{list}{$\cdot$}{\leftmargin=0em} 
% Compress items in list together 
   \itemsep -0.5em \vspace{-0.5em} 
  }{
  \end{list}
  \vspace{0.5em} % Some space after the list of bullet points
}
\begin{document}
\maketitle
\section{Skills}
\textbf{Technical Skills:}
C/C++, JavaScript/TypeScript, Python, HTML/CSS \\
\textbf{Tools:}
 AWS, Docker/Compose, Git/Github, \LaTeX, PostgreSQL, Linux \\
\textbf{Languages:}
English, Romanian %, French, Spanish
\vspace{-1.5em}
\section{Education}
\begin{customS}
    {Toronto Metropolitan University}
    {September 2022 - June 2026}
    {Candidate for BSc Honours, Computer Science}
    {Toronto, Canada}
\textbf{Relevant Coursework:} Computer Science I, Computer Science II, Ethics \\
\textbf{Awards:} Entrance Scholarship 
\end{customS}
\vspace{-1.5em}
\section{Experience}
\begin{customS}
    {Metropolitan Aerospace and Combustion Hub (MACH)}
    {September 2022 - Present}
    {Transfer \& Control Software Lead}
    {Toronto, Canada}
\item Experienced in coding with Arduino to control industrial or experimental processes.
\item Processed sensor data on a Linux system using a LabJack data acquisition unit and developing GUIs for data visualization. 
\item Continuously involved in team management, including task delegation, scheduling and running meetings.
\end{customS}
\begin{customS}
    {Robotics For All}
    {January 2020 - September 2022}
    {Software Development Team Member}
    {}
\item Initiated collaborations with school districts in Southern California via cold emailing.
\item Led the development of integrations within their Google Drive and Slack workspaces.
\item Gathered feedback and iterated on designs with volunteers to deliver high-quality software solutions.
\end{customS}
\vspace{-1.5em}
\section{Projects}
\begin{customS}
    {Substitute Teacher Contactor}
    {TypeScript, Node, Docker, Slack Bolt}
    {Robotics For All}
    {}
\item Developed a Slack application/integration that retrieves available volunteer teachers and coordinators from a Google Sheet using their Cloud API, simplifying the substitution process for teachers.
\item Managed all aspects of the system within Slack, without the need for further configuration.
\item Deployed the system using AWS ECR/ECS on an EC2 instance connected to an RDS database.
\end{customS}
\begin{customS}
    {Simple Notes}
    {Go, Postgres, Docker}
    {Collaborative Personal Project}
    {}
\item Designed and developed a simple notes application backend with a full CRUD API using GoFiber and PostgreSQL.
\item Utilized an ORM for general queries and raw SQL for custom and complex queries, ensuring efficient database operations.
\item Deployed the application on Heroku to enable access to users from anywhere, at any time.
\end{customS}
\vspace{-1.5em}
\end{document}
